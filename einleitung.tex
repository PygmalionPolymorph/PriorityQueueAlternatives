\section{Einleitung}
In diskreten eventbasierten Simulationen k�nnen \ac{PQ}s als Repr�sentation des \ac{PES} verwendet werden. Das \ac{PES} ist in solchen Simulationen die Abbildung der noch in der Zukunft liegenden Events als Menge. Die Events sind priorisierte Elemente dieser Menge. Das \ac{PES}-Problem bezieht sich auf die Performance der darunterliegenden Datenstruktur. Diese wirkt sich auf die Performance der ganzen Simulation aus, da bis zu 40\% der Rechenzeit f�r die Verwaltung des \ac{PES} aufgewendet wird. \cite{li_mlist} \\
Um die Auswirkung dieses Problems zu verringern, gibt es verschiedene Ans�tze, deren Effektivit�t von den Simulationsparametern, vor allem von der Anzahl der Events und der Verteilung der Priorit�ten derselben, abh�ngt.
\par
{\centering
Die zentrale Fragestellung lautet: \\
\textbf{Welche der betrachteten Datenstrukturen ist f�r die Optimierung einer diskreten eventbasierten Simulation  die beste Wahl?} \par}
\pagebreak
Im Rahmen dieser Arbeit ist mit ,,kleiner'' bzw. ,,gr��er'' immer ,,h�her Priorisiert'' resp. ,,niedriger Priorisiert'' gemeint.
Der Begriff der ,,amortisierten Zeit'' nach Sleator und Tarjan \cite{amortized-time} bezieht sich auf die tats�chlich erwartbare Laufzeitkomplexit�t bei Messung im Gegensatz zur theoretisch errechneten. 

Aufgrund des Rahmens dieser Arbeit ist der Vergleich auf \textit{\acs{PQ}, Parallel Heap, \acs{CQ}} und \textit{\acs{MList}} beschr�nkt. Als weitere, unbetrachtete Alternativen seien erw�hnt: \textit{SNOOPy CQ, DynamicList, LadderQueue, SplayTree, Pagoda, SkewHeap}. \cite{li_improved}\cite{snoopy}\cite{brown_cq}\cite{tang_ladder_2005}
