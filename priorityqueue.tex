\section{Priority Queues}
Eine \ac{PQ} ist ein abstrakter Datentyp, der auf einer Warteschlange basiert. Sie stellt die Operationen \textit{enqueue} und \textit{dequeue} zur Verf�gung, die Elemente zur Datenstruktur hinzuf�gen respektive entfernen und zur�ckgeben. Die Elemente in einer \textit{\ac{PQ}} werden mit einem Priorit�tswert versehen, die eine lineare Ordnung bilden. Beim Entfernen eines Elementes wird das Kleinste zur�ckgegeben. Neben der Nutzung als \ac{PES} kann die \textit{\ac{PQ}} beispielsweise im Dijkstra-Algorithmus oder zur Bandbreitenregelung in Netzwerken verwendet werden. \cite{neuhaus}
\subsection{Implementation}
Neben einer naiven Implementation der \ac{PQ} als \ac{LL}, die nach der Priorit�t der Elemente sortiert wird, gibt es zahlreiche andere M�glichkeiten. Eine M�gliche ist ein Heap, als Beispiel sei hier ein bin�rer Min-Heap dargestellt. Durch die notwendige Bedingung, dass die Kinder eines Knotens zu jeder Zeit gr��er sein m�ssen, als der Wert desselben, ist hier die n�tige lineare Ordnung bereits gegeben.
Das Einf�gen eines Elements in den Heap (\textit{enqueue}) geschieht durch Anh�ngen ans Ende des Baumes und anschlie�endem iterativen Vergleichen und Tauschen mit dem Elternelement, bis die Heapbedingung wiederhergestellt ist.
Um das kleinste Element zu erhalten (\textit{dequeue}), wird die Wurzel entfernt, zur�ckgegeben, und durch das letzte Element ersetzt und anschlie�end erneut durch wiederholtes Vergleichen und Tauschen des Elternknotens mit den Kindern die Heapbedingung wiederhergestellt. \cite{neuhaus}
\subsection{Performance}
Die \textit{dequeue} Operation kann bei einem bin�ren Heap im durchschnittlichen Fall in logarithmischer Zeit durchgef�hrt werden ($\Theta(log n)$), die \textit{enqueue} Operation im schlechtesten Fall ebenfalls logarithmisch ($O(log n)$). Initial ben�tigt der Heap allerdings einen Aufwand von $ O(n) $, um die Heapeigenschaft herzustellen. \cite{cormen2001introduction}
