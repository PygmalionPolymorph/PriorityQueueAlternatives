\section{Parallel Heap}
Als Alternative zum klassischen Heap gibt es f�r die parallele Verarbeitung mehrerer Events den \textit{Parallel Heap}.
Dieser ist im grunds�tzlichen Aufbau ein Heap, bei dem die Knoten Arrays der Gr��e $r$ sind, welche die Elemente halten. Entsprechend befindet sich das Element \textit{i} an der Stelle $(i-( \lceil \frac{i}{r} \rceil -1)*r)$  des Arrays, in Knoten $ \lceil \frac{i}{r} \rceil $.
\cite{parallelHeap-structure}\cite{prasad_parallel_1995}

\subsection{Dequeue}
Beim L�schen wird der Wurzelknoten entfernt. Da dies ein Array aus \textit{r} Elementen ist, k�nnen \textit{r} Prozessorkerne ausgelastet werden, um mit diesen Daten zu arbeiten.
Der freigewordene Platz wird mit dem gr��ten Knoten aufgef�llt. Sollte dieser aus weniger als \textit{r} Elementen bestehen, wird dies mit den gr��ten Elementen des zweitgr��ten Knotens aufgef�llt. \cite{parallelHeap-structure}\cite{prasad_parallel_1995}

Wie bei gew�hnlichen Heaps auch, muss nun die Heapeigenschaft wiederhergestellt werden. Das Vorgehen ist dem Prinzip nach gleich; mit kleinen Erweiterungen.
Die Elemente des aktuellen Knotens und seiner beiden Kinder werden zusammengef�hrt und sortiert. Die kleinsten \textit{r} Elemente bleiben im aktuellen Knoten. \cite{parallelHeap-structure}\cite{prasad_parallel_1995}

Sei \textit{x} die Priorit�t des gr��ten Elementes im linken Kind und \textit{y} die des Rechten.
Wenn $x \geq y$ ist, werden die \textit{r} kleinsten Elemente im linken Kind untergebracht, der Rest im rechten Kind. Da die Heapeigenschaft f�r das linke Kind erf�llt ist, wird nachfolgend nur noch der rechte Teilbaum betrachtet; das rechte Kind wird zum aktuellen Knoten.
Die genannten Schritte werden wiederholt, bis der aktuelle Knoten ein Blatt ist, oder f�r diesen die Heapeigenschaft erf�llt ist.
Falls $x < y$ ist, werden die Rollen von linkem und rechtem Kind vertauscht.  \cite{parallelHeap-structure}\cite{prasad_parallel_1995}

Aufgrund des beschriebenen Mechanismus ist die Verarbeitung mehrerer Elemente parallel, wodurch eine deterministische Abfolge nicht vollst�ndig garantiert ist. Soll der Parallel Heap f�r \acs{DES} genutzt werden, muss dies an anderer Stelle sichergestellt werden. \cite{PDES}

\subsection{Enqueue}
Wenn \textit{n} Elemente in den Heap eingef�gt werden sollen, werden diese zuerst sortiert und danach durch den \textit{insert-update}-Prozess eingef�gt:
Der Prozess beginnt immer mit dem Wurzelknoten und endet am Zielknoten.
Zielknoten ist der Letzte, sofern dieser die einzuf�genden Elemente aufnehmen kann, sonst ein neuer Knoten, welcher nach dem aktuell Letzten eingef�gt wird.
Wenn im Zielknoten $k < n$ Elemente vorhanden sind, m�ssen zwei Zyklen des Prozesses durchlaufen werden. In dem ersten werden die $n-k$ kleinsten Elemente einsortiert, im Zweiten der Rest.  \cite{parallelHeap-structure}\cite{prasad_parallel_1995}

Folgende Schritte werden wiederholt, bis der Zielknoten erreicht ist. Im Zielknoten werden schlie�lich alle Elemente untergebracht werden k�nnen.
Die \textit{n} einzuf�genden Elemente werden mit denen des aktuellen Knotens zusammengef�hrt und sortiert.
Anschlie�end werden die \textit{r} kleinsten Elemente im aktuellen Knoten belassen; alle Verbleibenden werden zu dem Kind mitgenommen, welches auf dem Weg zum Zielknoten liegt. \cite{parallelHeap-structure}\cite{prasad_parallel_1995}
